% COMSOL 微波電漿研究簡報(精簡版 + 理論推導)
% 風格:極簡黑白
% 總頁數:約 15 張

\documentclass[11pt, aspectratio=169]{beamer}
\usetheme[progressbar=frametitle, block=transparent]{metropolis}
\usefonttheme{professionalfonts}

% ========== 字體設定 ==========
\usefonttheme[onlymath]{serif}
\usepackage{fontspec}
\usepackage{xeCJK}
\setmainfont{Times New Roman}
\setsansfont{Times New Roman}
\setCJKmainfont{Songti TC}
\setCJKsansfont{Songti TC}
\setCJKmonofont{Songti TC}
\usepackage{newtxmath}
\setmonofont{Courier New}

% ========== 數學與圖表 ==========
\usepackage{amsmath, amssymb}
\usepackage{graphicx}
\usepackage{booktabs}
\usepackage[backend=biber,style=authoryear]{biblatex}
\addbibresource{../references.bib}

% ========== 配色方案:極簡黑白 ==========
\setbeamercolor{frametitle}{fg=black, bg=white}
\setbeamercolor{progress bar}{fg=black}
\setbeamercolor{normal text}{fg=black!90, bg=white}
\setbeamercolor{block title}{fg=black, bg=black!5}
\setbeamercolor{block body}{fg=black!90, bg=white}
\setbeamercolor{alerted text}{fg=black}
\setbeamercolor{itemize item}{fg=black}
\setbeamercolor{itemize subitem}{fg=black!70}
\setbeamercolor{enumerate item}{fg=black}

% ========== 移除導覽符號 ==========
\setbeamertemplate{navigation symbols}{}

% ========== 標題資訊 ==========
\title{COMSOL 模擬 2.45GHz 圓形波導\\微波電漿之參數研究}
\subtitle{不同功率下激發模式轉變分析}
\author{陳亮宇}
\institute{國立新竹科學園區實驗高級中等學校}
\date{2025 年 10 月}

% ========== 正文開始 ==========
\begin{document}

% ========================================
% 第一部分:導言 (3 張)
% ========================================

\maketitle

\begin{frame}[t]{研究背景與動機}
  \small
  \begin{columns}[T]
    \begin{column}{0.48\textwidth}
      \textbf{應用背景}
      \begin{itemize}
        \item 微波電漿具備高密度、低污染特性,廣泛應用於蝕刻、CVD 與薄膜沉積等半導體製程。\cite{comsol_blog,chen2016}
        \item 2.45\,GHz 系統的截止電子密度約為 $7.4\times10^{16}\,\mathrm{m}^{-3}$,超過後理應反射入射微波。\cite{griffiths2017,comsol_blog}
      \end{itemize}

      \medskip

      \textbf{知識落差}
      \begin{itemize}
        \item 過密度電漿仍能持續吸收能量,顯示存在體積波以外的耦合機制。\cite{swp_wiki,nagatsu_sugai}
        \item 既有冪次律與幾何分析多侷限於低功率單一條件,缺乏跨功率/半徑的統一圖譜。\cite{per_electron_power,below_cutoff}
      \end{itemize}
    \end{column}

    \begin{column}{0.48\textwidth}
      \textbf{研究動機}
      \begin{itemize}
        \item 釐清體積波與表面波加熱的轉換條件,理解能量如何在過密度區維持。\cite{swp_wiki}
        \item 量化功率、半徑與電子密度之間的關聯,建立可供設備調校的行為地圖。
      \end{itemize}

      \medskip

      \textbf{研究策略}
      \begin{itemize}
        \item 建立 COMSOL 耦合電磁與電漿模組,橫跨 $P_{\text{in}}=1\sim10^5$ W 與廣泛腔體半徑掃描。
        \item 以資料管線萃取眾數密度、臨界功率與模式邊界,對照理論模型驗證。
      \end{itemize}
    \end{column}
  \end{columns}
\end{frame}

\begin{frame}[t]{三個核心研究問題}
  \small
  \begin{block}{研究問題 1:功率-密度關係}
    輸入功率 $P_{\text{in}}$ 如何影響電子密度 $n_e$?是否存在冪次律關係?
    \[
    n_e \propto P_{\text{in}}^{\alpha}
    \]
  \end{block}

  \smallskip

  \begin{block}{研究問題 2:幾何效應}
    不同腔體半徑 $r$ 如何影響點火功率與電漿徑向尺度?
  \end{block}

  \smallskip

  \begin{block}{研究問題 3:激發模式轉換}
    激發模式轉換的臨界功率與腔體半徑的關係?物理機制為何?
  \end{block}
\end{frame}

% ========================================
% 第二部分:研究方法 (2 張)
% ========================================

\begin{frame}[t]{COMSOL 模擬設置:幾何與條件}
  \small
  \begin{columns}[T]
    \begin{column}{0.55\textwidth}
      \textbf{幾何與邊界}
      \begin{itemize}
        \item 2D 軸對稱模型,左端 TEM 端口饋入;同軸內外邊界視為理想導體。
        \item 內壁允許電子表面複合,離子以玻姆速度入射後中和。
      \end{itemize}
      \medskip
      \textbf{氣體與物種}
      \begin{itemize}
        \item 氬氣:$p = 60$ mTorr,$T_0 = 400$ K。
        \item 物種 $\{e^-, \mathrm{Ar}, \mathrm{Ar}_s, \mathrm{Ar}^+\}$;碰撞截面採 LXCat Phelps (2017),含激發、逐步電離、潘寧電離與壁面淬滅。
      \end{itemize}
    \end{column}

    \begin{column}{0.41\textwidth}
      \begin{center}
        \includegraphics[width=0.95\textwidth]{../個別科學研究成果報告書/plots/模型設置.png}
      \end{center}
      \vspace{-0.4em}
      {\scriptsize 軸對稱幾何:同軸饋入/內外邊界理想導體。}
    \end{column}
  \end{columns}
\end{frame}

\begin{frame}[t]{COMSOL 模擬設置:方程與參數掃描}
  \small
  \begin{columns}[T]
    \begin{column}{0.5\textwidth}
      \textbf{統御方程與耦合}
      \medskip
      \begin{itemize}
        \item \textbf{電磁場(EMW):}
        \item[]
          \[
            \nabla \times (\mu_r^{-1}\nabla \times \mathbf{E}) - k_0^2\!\left(\epsilon_r - \frac{j\sigma}{\omega\epsilon_0}\right)\mathbf{E}=0
          \]
          由電漿導電率 $\sigma$ 與介電常數回饋。
        \item \textbf{電子連續性(Plas):}
        \item[]
          \[
            \partial_t n_e + \nabla\cdot\boldsymbol{\Gamma}_e = R_e
          \]
          $R_e$ 含激發、逐步電離、潘寧電離與表面損失;能量方程決定電子溫度與 $\sigma$。
      \end{itemize}
    \end{column}
    \begin{column}{0.46\textwidth}
      \textbf{掃描策略與輸出}
      \medskip
      \begin{itemize}
        \item \textbf{功率掃描}:$r=47.7$ mm,$P_{in}=1\sim10^5$ W;蒐集電子密度、功率沉積與反射功率。
        \item \textbf{幾何掃描}:$r=5\sim1000$ mm,基準 $P_{in}=20$ W,逐步調升以取得最小點火功率 $P_{cutoff}(r)$。
        \item \textbf{模擬輸出}:儲存 2D 場量、徑向切片與 sweep log,供後續結果重現與統計分析。
      \end{itemize}
    \end{column}
  \end{columns}
\end{frame}

\begin{frame}[t]{數據分析方法}
  \footnotesize
  \begin{columns}[T]
    \begin{column}{0.48\textwidth}
      \textbf{KDE 眾數計算}

      在對數空間進行核密度估計:
      \[
      \text{KDE}(x) = \frac{1}{nh} \sum_{i=1}^n K\left(\frac{x - x_i}{h}\right)
      \]

      高斯核函數:
      \[
      K(u) = \frac{1}{\sqrt{2\pi}} e^{-\frac{u^2}{2}}
      \]

      優勢:抗極端值干擾
    \end{column}

    \begin{column}{0.48\textwidth}
      \textbf{Log-Log 線性擬合}

      對數空間線性回歸:
      \[
      \log_{10} n_e = \alpha \log_{10} P_{\text{in}} + \log_{10} C
      \]

      求解參數:
      \begin{itemize}
        \item 冪次指數 $\alpha$
        \item 係數 $C$
        \item 決定係數 $R^2$
      \end{itemize}

      \smallskip

      \textbf{實作細節}:所有數據分析腳本與資料已收錄於附件 `code/` 與 `research*/data/`
    \end{column}
  \end{columns}
\end{frame}

% ========================================
% 第三部分:核心結果 (6 張)
% ========================================

\begin{frame}[t]{結果 1:功率-密度冪次律關係}
  \small
  \begin{columns}[T]
    \begin{column}{0.42\textwidth}
      \textbf{主要發現}
      \[
      n_e \propto P_{\text{in}}^{0.73\sim0.78}
      \]

      \begin{itemize}
        \item 對數-對數擬合的斜率穩定落在 0.73–0.78
        \item 次線性冪律顯示功率提升時電子密度成長趨緩
        \item 觀察與數據涵蓋 $P_{in}=1\sim 10^5$ W
      \end{itemize}

    \end{column}

    \begin{column}{0.56\textwidth}
      \begin{center}
        \includegraphics[width=\textwidth]{../個別科學研究成果報告書/plots/P_in vs. Electron Density 的 log-log 圖.png}
      \end{center}
    \end{column}
  \end{columns}
\end{frame}

\begin{frame}[t]{結果 1:點火閾值視覺證據}
  \small
  \begin{columns}[T]
    \begin{column}{0.48\textwidth}
      \textbf{電子密度分布}
      \begin{itemize}
        \item 低功率掃描中,密度機率分佈出現明顯斷層
        \item 斷層對應輸入功率跨越點火閾值 $E_{crit}$
        \item 支持「雪崩式擊穿」:閾值前難以維持電漿,閾值後密度跳升
      \end{itemize}
    \end{column}

    \begin{column}{0.48\textwidth}
      \begin{center}
        \includegraphics[width=\textwidth]{../個別科學研究成果報告書/plots/Electron Density Distribution 顯示低功率斷層.png}
      \end{center}
      \vspace{-0.4em}
      {\scriptsize 電子密度機率分佈:10–40 W 區間出現斷層,標記點火跳變。}
    \end{column}
  \end{columns}
\end{frame}

\begin{frame}[t]{結果 1:理論解析}
  \small
  \begin{columns}[T]
    \begin{column}{0.48\textwidth}
      \textbf{粒子與能量平衡}
      \begin{itemize}
        \item 穩態條件:$G=L_{diff}+L_{vol}$,結合能量守恆得到
        \[
        P_{in}=C_1 \bar{n}_e + C_2 \bar{n}_e^2
        \]
        \item $C_1=\dfrac{D_a V \mathcal{E}_c}{\Lambda^2}$(擴散損失)、$C_2=\beta_{rec} V \mathcal{E}_c$(體複合損失)
      \end{itemize}
    \end{column}

    \begin{column}{0.48\textwidth}
      \textbf{冪次律的意義}
      \begin{itemize}
        \item 低密度:$C_1 \bar{n}_e \gg C_2 \bar{n}_e^2 \Rightarrow \bar{n}_e \propto P_{in}$
        \item 高密度:$C_2 \bar{n}_e^2 \gg C_1 \bar{n}_e \Rightarrow \bar{n}_e \propto P_{in}^{1/2}$
        \item 實際範圍的 $\alpha = 0.73\text{--}0.78$ 顯示兩種損失通道同時主導
      \end{itemize}
    \end{column}
  \end{columns}
\end{frame}

\begin{frame}[t]{結果 1:小結}
  \small
  \begin{itemize}
    \item $n_e$ 與 $P_{in}$ 呈次線性冪律,功率提升時密度增長趨於飽和。
    \item 擴散與體複合損失並存,限制能量耦合效率並決定冪次落在 $0.73\text{--}0.78$。
  \end{itemize}
\end{frame}

\begin{frame}[t]{結果 2:模式轉換能量沉積(低功率段)}
  \small
  \begin{columns}[T]
    \begin{column}{0.44\textwidth}
      \textbf{體積波跡象}
      \begin{itemize}
        \item 10 W:功率沉積集中於腔體中心,維持體積波加熱。
        \item 20 W:壁面附近出現能量帶,顯示進入點火過渡。
      \end{itemize}
    \end{column}

    \begin{column}{0.52\textwidth}
      \begin{center}
        \begin{minipage}{0.48\textwidth}
          \includegraphics[width=\textwidth]{../個別科學研究成果報告書/plots/10W功率沉積.png}\\[-0.2em]
          {\scriptsize 10 W(體積波)}
        \end{minipage}\hfill
        \begin{minipage}{0.48\textwidth}
          \includegraphics[width=\textwidth]{../個別科學研究成果報告書/plots/20W功率沉積.png}\\[-0.2em]
          {\scriptsize 20 W(過渡)}
        \end{minipage}
      \end{center}
    \end{column}
  \end{columns}
\end{frame}

\begin{frame}[t]{結果 2:模式轉換能量沉積(高功率段)}
  \small
  \begin{columns}[T]
    \begin{column}{0.44\textwidth}
      \textbf{表面波成形}
      \begin{itemize}
        \item 50 W:能量帶貼近介電壁面,表面波逐漸主導。
        \item 200 W:沉積壓縮於介面薄層,轉為穩定表面波加熱。
      \end{itemize}
    \end{column}

    \begin{column}{0.52\textwidth}
      \begin{center}
        \begin{minipage}{0.48\textwidth}
          \includegraphics[width=\textwidth]{../個別科學研究成果報告書/plots/50W功率沉積.png}\\[-0.2em]
          {\scriptsize 50 W(過渡)}
        \end{minipage}\hfill
        \begin{minipage}{0.48\textwidth}
          \includegraphics[width=\textwidth]{../個別科學研究成果報告書/plots/200W功率沉積.png}\\[-0.2em]
          {\scriptsize 200 W(表面波)}
        \end{minipage}
      \end{center}
    \end{column}
  \end{columns}
\end{frame}

\begin{frame}[t]{結果 2:理論解析}
  \small
  \begin{columns}[T]
    \begin{column}{0.48\textwidth}
      \textbf{模式轉換條件}
      \begin{itemize}
        \item 體積波階段:中心區域吸收為主,電子密度維持在 $n_e < n_{cr}$。
        \item 過渡階段:局部密度逼近 $n_{cr}$ 時,靠近介面的電場增強形成表面能量帶。
        \item 表面波階段:$n_e > n_{cr}$ 後,能量侷限於介電壁附近的趨膚深度 $\delta_p$。
      \end{itemize}
    \end{column}

    \begin{column}{0.48\textwidth}
      \textbf{能量沉積平衡}
      \begin{itemize}
        \item 體積波:吸收體積與損失(擴散、體複合)競逐,能量分布寬而均勻。
        \item 表面波:能量沿介面傳輸,沉積厚度取決於 $\delta_p$ 與擴散長 $L_{phy}$。
        \item 功率提升將沉積由中心推向壁面,對應圖示中的 10/20/50/200 W 演化。
      \end{itemize}
    \end{column}
  \end{columns}
\end{frame}

\begin{frame}[t]{結果 2:小結}
  \small
  \begin{itemize}
    \item 低功率維持體積波,能量沉積集中於腔體中心且電子密度低於 $n_{cr}$。
    \item 功率逐步提升時,介面附近的能量帶增強,顯示系統進入表面波過渡。
    \item 表面波形成後,能量侷限於介電壁附近,後續功率主要調整表面層強度。
  \end{itemize}
\end{frame}

\begin{frame}[t]{結果 3:徑向尺度飽和現象}
  \small
  \begin{columns}[T]
    \begin{column}{0.48\textwidth}
      \textbf{兩段式行為}
      \begin{itemize}
        \item $r\lesssim 200$ mm:電漿衰減半徑幾乎與腔體半徑線性同步(壁面限制)
        \item $r\gtrsim 200$ mm:衰減半徑趨於平台值,幾何放大不再增加電漿尺度
        \item 飽和值由表面波趨膚深度與擴散-複合長度共同決定
      \end{itemize}

      \smallskip
    \end{column}

    \begin{column}{0.48\textwidth}
      \begin{center}
        \includegraphics[width=\textwidth]{../個別科學研究成果報告書/plots/徑向切片分析圖.png}
      \end{center}
    \end{column}
  \end{columns}
\end{frame}

\begin{frame}[t]{結果 3:物理解釋}
  \small
  \begin{columns}[T]
    \begin{column}{0.48\textwidth}
      \textbf{擴散主導區 ($r \lesssim 200$ mm)}
      \begin{itemize}
        \item 粒子尚未在體積內複合便抵達管壁,$\beta_{rec}$ 可忽略
        \item 徑向解受邊界條件限制,$R_{\text{plasma}} \approx r$
        \item 電漿填滿腔體,徑向尺度跟隨幾何放大
      \end{itemize}
    \end{column}

    \begin{column}{0.48\textwidth}
      \textbf{複合主導區 ($r \gtrsim 200$ mm)}
      \begin{itemize}
        \item 定義物理尺度 $L_{phy} = \sqrt{D_a / (\beta_{rec} \bar{n}_e)}$
        \item $r \gg L_{phy}$ 時,粒子在體內複合,與管壁脫離
        \item 表面波加熱厚度 $\delta_p$ 與 $L_{phy}$ 共同設定飽和值
      \end{itemize}
    \end{column}
  \end{columns}
\end{frame}

\begin{frame}[t]{結果 3:徑向衰減量化}
  \small
  \begin{columns}[T]
    \begin{column}{0.48\textwidth}
      \textbf{衰減半徑指標}
      \begin{itemize}
        \item 取電子密度衰減至 $\alpha=0.1,0.5,0.9$ 的半徑作為尺度指標
        \item 小腔體:指標幾乎貼合 $y=x$,證實壁面限制
        \item 大腔體:三條曲線都趨於常值,顯示與壁面脫離
      \end{itemize}
    \end{column}

    \begin{column}{0.48\textwidth}
      \begin{center}
        \includegraphics[width=\textwidth]{../個別科學研究成果報告書/plots/分析電子密度的徑向衰減特性.png}
      \end{center}
      \vspace{-0.4em}
      {\scriptsize 電漿衰減半徑 vs. 腔體半徑:小腔體貼合 $y=x$,大腔體進入飽和值。}
    \end{column}
  \end{columns}
\end{frame}

\begin{frame}[t]{結果 3:小結}
  \small
  \begin{itemize}
    \item 小腔體內徑向半徑與腔體半徑同步,顯示壁面限制的擴散主導。
    \item 大腔體區域出現飽和值,粒子在體內複合並與壁面脫離。
    \item 表面波趨膚深度與 $L_{phy}$ 決定了 W-Mode 下的內在尺度上限。
  \end{itemize}
\end{frame}

\begin{frame}[t]{結果 4:幾何參數對點火邊界的影響}
  \small
  \begin{columns}[T]
    \begin{column}{0.42\textwidth}
      \textbf{觀察重點}
      \begin{itemize}
        \item $r<35.9$ mm 的案例需要遠高於其他半徑的點火功率,因為波導落在截止區,微波只能以漸逝波衰減前進。
        \item 在高功率區段,正常傳播區 ($r>35.9$ mm) 的曲線收斂,顯示表面波加熱時幾何影響趨於消失。
        \item 特定半徑(如 $r=500$ mm)出現雙階段躍升,推測依序跨越點火閾值與模式轉換兩個門檻。
      \end{itemize}
    \end{column}

    \begin{column}{0.52\textwidth}
      \begin{center}
        \includegraphics[width=\textwidth]{../個別科學研究成果報告書/plots/r2資料集比較圖.png}
      \end{center}
    \end{column}
  \end{columns}
\end{frame}

\begin{frame}[t]{結果 4:激發模式轉換臨界功率}
\framesubtitle{點火功率擬合}
  \small
  \begin{columns}[T]
    \begin{column}{0.48\textwidth}
      \textbf{點火功率擬合}
      \[
      P_{cutoff} \propto r^{-7.42}
      \]

      \begin{itemize}
        \item 將波導截止範圍 ($r<35.9$ mm) 的資料擬合後得到 $P_{cutoff} \propto r^{-7.42}$,量化小半徑點火成本的急遽提升。
        \item 臨界半徑 $r_c = 35.9$ mm 對應 $TE_{11}$ 截止,與前一頁觀察的族群邊界一致。
        \item 在 log--log 座標下,該冪次量測的是指數模型在此半徑範圍內的局部斜率,便於與理論與實驗交叉驗證。
      \end{itemize}
    \end{column}

    \begin{column}{0.48\textwidth}
      \begin{center}
        \includegraphics[width=\textwidth]{../個別科學研究成果報告書/plots/截止功率-半徑趨勢分析.png}
      \end{center}
    \end{column}
  \end{columns}
\end{frame}

\begin{frame}[t]{結果 4:激發模式轉換臨界功率}
\framesubtitle{物理圖像}
  \small
  \begin{columns}[T]
    \begin{column}{0.52\textwidth}
      \textbf{物理圖像}
      \begin{itemize}
        \item 小半徑:波導位於截止區,場型為漸逝波,入射功率必須放大才可能在腔體內達到 $E_{crit}$。
        \item 大半徑:遠離截止可直接傳播;在表面波主導區,正常傳播區的曲線收斂,加熱機制與幾何尺寸脫鉤。
        \item 點火但 $n_e < n_{cr}$ 時維持體積波加熱;當 $n_e \ge n_{cr}$ 則進入表面波主導。
      \end{itemize}
    \end{column}

    \begin{column}{0.44\textwidth}
      \begin{center}
        \includegraphics[width=\textwidth]{../個別科學研究成果報告書/plots/截止功率-半徑趨勢分析.png}
      \end{center}
    \end{column}
  \end{columns}
\end{frame}

\begin{frame}[t]{結果 4:理論解析}
  \small
  \begin{columns}[T]
    \begin{column}{0.48\textwidth}
      \begin{block}{漸逝波點火}
        \footnotesize
        \begin{itemize}
          \item \textbf{波導色散}:決定傳播或截止
            \[
              \beta^2 = \left(\frac{\omega}{c}\right)^2 - \left(\frac{\chi'_{11}}{r}\right)^2
            \]
          \item \textbf{衰減常數}:$r<r_c$ 時 $\beta=i\alpha$
            \[
              \alpha(r) = \sqrt{\left(\frac{\chi'_{11}}{r}\right)^2 - \left(\frac{\omega}{c}\right)^2}
            \]
          \item \textbf{場強放大}:漸逝波需放大才能點火
            \[
              E_{in}(r) = E_{crit} e^{\alpha(r) L}
            \]
          \item \textbf{點火功率模型}:入口功率隨 $\alpha(r)$ 指數成長
            \[
              P_{cutoff}(r) = E_{crit}^2 \exp\!\bigl(2L\,\alpha(r)\bigr)
            \]
        \end{itemize}
      \end{block}
    \end{column}

    \begin{column}{0.48\textwidth}
      \begin{block}{對數斜率與偏差}
        \footnotesize
        \begin{align*}
          k(r) &= -\frac{d \ln P_{cutoff}}{d \ln r} =
          \frac{2L(\chi'_{11})^2}{r^2 \sqrt{\left(\frac{\chi'_{11}}{r}\right)^2 - \left(\frac{\omega}{c}\right)^2}}
        \end{align*}
        \begin{itemize}
          \item 取 $L_{\text{eff}}=20$ mm 時,$k(r)$ 在 5--30 mm 區間落於 $\sim4$--$15$,平均約 7.4,與前一頁冪次擬合一致。
          \item 偏差來源:$L_{\text{eff}}$ 估計、點火後密度回饋、半徑採樣間距。
          \item 理論與模擬皆顯示小半徑點火成本呈指數級上升。
        \end{itemize}
      \end{block}
    \end{column}
  \end{columns}
\end{frame}

\begin{frame}[t]{結果 4:小結}
  \small
  \begin{itemize}
    \item 點火功率隨半徑減小呈急劇上升,反映波導截止造成的漸逝波衰減。
    \item 當功率足以突破臨界密度後,正常傳播區的曲線收斂,顯示表面波加熱的普適耦合效率。
    \item 擬合冪次與理論指數關係一致,證實模型對幾何縮尺度的敏感性。
  \end{itemize}
\end{frame}

\begin{frame}[t]{綜合討論:三項核心機制}
  \small
  \begin{block}{機制 1:點火閾值 $E_{crit}$}
    功率不足時電場無法達到擊穿閾值,電漿停留在未點火區;閾值跨越後電子密度出現跳變。
  \end{block}

  \vspace{0.2cm}

  \begin{block}{機制 2:模式轉換 $n_e \gtrsim n_{cr}$}
    當電子密度超過臨界值,體積波被截止並改由表面波維持,加熱區域壓縮至趨膚深度。
  \end{block}

  \vspace{0.2cm}

  \begin{block}{機制 3:幾何限制與飽和}
    波導截止半徑 $r_c = 1.841\,c/\omega$ 決定點火難度;大半徑區則受擴散-複合內在尺度支配而飽和。
  \end{block}

  \vspace{0.2cm}

\begin{alertblock}{整合結論}
  點火閾值、臨界密度與幾何尺度共同勾勒出三個運行區域,解釋功率-密度map的所有趨勢。
\end{alertblock}
\end{frame}

\begin{frame}[t]{綜合討論:運行區域地圖}
  \small
  \begin{columns}[T]
    \begin{column}{0.46\textwidth}
      \textbf{三個運行區}
      \begin{itemize}
        \item 區域 I:點火且 $n_e>n_{cr}$,表面波主導高密度區
        \item 區域 II:點火但 $n_e<n_{cr}$,維持體積波加熱
        \item 區域 III:功率不足,停留在未點火狀態
      \end{itemize}

      \smallskip

      \textbf{邊界意義}
      \begin{itemize}
        \item 紅虛線:$E_{crit}$ 導出的 $P_{cutoff}(r)$,決定能否點火
        \item 藍虛線:$n_e=n_{cr}$,決定體積波或表面波機制
      \end{itemize}
    \end{column}

    \begin{column}{0.5\textwidth}
      \begin{center}
        \includegraphics[width=\textwidth]{../個別科學研究成果報告書/plots/區域解析.png}
      \end{center}
      \vspace{-0.4em}
      {\scriptsize 功率–密度圖:紅、藍邊界將運行條件劃分為三個區域。}
    \end{column}
  \end{columns}
\end{frame}

% ========================================
% 第五部分:結論與展望 (2 張)
% ========================================

\begin{frame}[t]{主要貢獻}
\framesubtitle{冪次律與物理解耦}
  \small
  \begin{block}{1. 功率-密度冪次律}
    在 $P_{in}=1\sim 10^5$ W 範圍量化 $n_e \propto P_{in}^{0.7-0.8}$,並以粒子/能量平衡模型解釋其物理來源。
  \end{block}

  \smallskip

  \begin{block}{2. 點火閾值與模式轉換解耦}
    找出點火閾值($E_{crit}$)與模式轉換($n_{cr}$)兩條邊界,建構三區域運行地圖。
  \end{block}
\end{frame}

\begin{frame}[t]{主要貢獻}
\framesubtitle{資料管線與幾何效應}
  \small
  \begin{block}{3. 建立完整數據分析管線}
    \begin{itemize}
      \item VTU 檔案解析 → 統計聚合 → 視覺化
      \item KDE 方法在對數空間的優越性
      \item 所有腳本與資料夾收錄於附件,支援結果重現
    \end{itemize}
  \end{block}

  \smallskip

  \begin{block}{4. 幾何效應與徑向飽和}
    說明波導截止造成的 $P_{cutoff}$ 急遽上升,以及大腔體下電漿徑向尺度的飽和值。
  \end{block}
\end{frame}

\begin{frame}[t]{未來工作}
  \small
  \textbf{後續工作重點}
  \begin{itemize}
    \item 在模型中納入多組分氣體與反應動力學,評估化學能量損失角色。\cite{fridman2008}
    \item 加入外加磁場並探討 ECR(電子迴旋共振)對點火與模式邊界的影響。\cite{lieberman2005,chabert2011}
  \end{itemize}
\end{frame}

% ========================================
% 結束頁
% ========================================

\begin{frame}[allowframebreaks]{參考文獻}
  \tiny
  \printbibliography
\end{frame}

\end{document}
